\kurzfassung

%% deutsch
\paragraph*{}
%In der Kurzfassung soll in kurzer und pr�gnanter Weise der wesentliche Inhalt der Arbeit beschrieben werden. Dazu z�hlen vor allem eine kurze Aufgabenbeschreibung, der L�sungsansatz sowie die wesentlichen Ergebnisse der Arbeit. Ein h�ufiger Fehler f�r die Kurzfassung ist, dass lediglich die Aufgabenbeschreibung (d.h. das Problem) in Kurzform vorgelegt wird. Die Kurzfassung soll aber die gesamte Arbeit widerspiegeln. Deshalb sind vor allem die erzielten Ergebnisse darzustellen. Die Kurzfassung soll etwa eine halbe bis ganze DIN-A4-Seite umfassen.

%Hinweis: Schreiben Sie die Kurzfassung am Ende der Arbeit, denn eventuell ist Ihnen beim Schreiben erst vollends klar geworden, was das Wesentliche der Arbeit ist bzw. welche Schwerpunkte Sie bei der Arbeit gesetzt haben. Andernfalls laufen Sie Gefahr, dass die Kurzfassung nicht zum Rest der Arbeit passt.



Diese Arbeit befasst sich mit der Frage, auf welche Weise ein Information Retrieval System realsiert werden kann, welches ein Archiv mit semistrukturierten Dokumenten, die sowohl Freitext als auch Metadaten enthalten, effizient nach vom Nutzer festgelegten und logisch verkn�pfbaren Kriterien durchsucht. Ziel der Implementierung ist es, dem Nutzer nach Abschluss des Suchvorgangs relevante Informationen zur�ckzuleifern und diese auf eine �bersichtliche Weise zu pr�sentieren. \\
Zun�chst wird die Problemstellung im Detail erl�utert, damit der Leser eine genaue Vorstellung �ber die Anforderungen, welche die Implementierung erf�llen soll, bekommt. Anschlie�end wird Information Retrieval im Allgemeinen vorgestellt, um dem Leser einen �berblick �ber die Thematik zu verschaffen. \\
Es folgt eine Vorstellung der beiden klassischen Information Retrieval Modelle boolesches Retrieval und Vektorraummodell inklusive Erl�uterung der Funktionsweise. Anschlie�end wird beschrieben, wie sich solche Modelle in Hinsicht auf deren Qualit�t bewerten und vergleichen lassen.\\
Nach dem Vermitteln der notwendigen theoretischen Kenntnisse beschreibt der Implementierungsteil, auf welche Weise und in welchen Bereichen die beiden Verfahren f�r die Implementierung dieser Arbeit zum Einsatz kamen und inwieweit eine Modifizierung zur Anpassung auf die vorliegende Problemstellung erfolgte. 
Neben der internen Funktionsweise wird in dieser Ausarbeitung auch die Benutzung der Oberfl�che erl�utert, um den Anwender mit der Bedienung des Systems vertraut zu machen.
Es folgt eine Zusammenfassung der Arbeit mit abschlie�ender Bewertung der Ergebnisse sowie ein Ausblick auf weitere M�gliche Verbesserungen.

\paragraph*{}
This paper discusses the question of how an information retrieval system that searches efficiently in semistructured data containing free text and metadata can be realized. The  search criteria for this system are determined by the user and can be freely connected with boolean operators. The aim of the implementation is to present relevant information in an clearly arranged way after finishing the search process. \\
First the problem is discussed in detail to give the reader a precise idea of the requirements which the implementation has to fit. Afterwards, information retrieval in general is presented to give the reader an overview on the topic.\\
The chapter about information retrieval is followed by a presentation of two classical information retrieval model including their functionality called boolean retrieval and vectorspace model. Afterwards, it is discussed how the quality of such systems can be estimated and compared. \\
After having provided the necessary theory, the paper continues with the implementation part, which explains which models were used in which way in this implementation and if they have been modified in any way to fit this problem. \\
Aside from the intern functionality, the reader learns about how to operate with the system via the given user interface. \\
The paper concludes with a summary including a final result evaluation and a presentation of future prospects for possible system improvements.




