\kurzfassung

%% deutsch
\paragraph*{}
%In der Kurzfassung soll in kurzer und pr�gnanter Weise der wesentliche Inhalt der Arbeit beschrieben werden. Dazu z�hlen vor allem eine kurze Aufgabenbeschreibung, der L�sungsansatz sowie die wesentlichen Ergebnisse der Arbeit. Ein h�ufiger Fehler f�r die Kurzfassung ist, dass lediglich die Aufgabenbeschreibung (d.h. das Problem) in Kurzform vorgelegt wird. Die Kurzfassung soll aber die gesamte Arbeit widerspiegeln. Deshalb sind vor allem die erzielten Ergebnisse darzustellen. Die Kurzfassung soll etwa eine halbe bis ganze DIN-A4-Seite umfassen.

%Hinweis: Schreiben Sie die Kurzfassung am Ende der Arbeit, denn eventuell ist Ihnen beim Schreiben erst vollends klar geworden, was das Wesentliche der Arbeit ist bzw. welche Schwerpunkte Sie bei der Arbeit gesetzt haben. Andernfalls laufen Sie Gefahr, dass die Kurzfassung nicht zum Rest der Arbeit passt.



Diese Arbeit befasst sich mit der Konzeption und Realisierung eines Information-Retrieval-Systems, welches ein Archiv mit semistrukturierten Dokumenten, die sowohl Freitext als auch Metadaten enthalten, effizient nach vom Nutzer festgelegten und logisch verkn�pfbaren Kriterien durchsucht. Ziel der Implementierung ist es, dem Nutzer nach Abschluss des Suchvorgangs zu seinem Informationsbed�rfnis passende Dokumente zur�ckzuliefern und diese auf eine �bersichtliche Weise zu pr�sentieren. \\
Zun�chst wird die Problemstellung im Detail erl�utert, damit der Leser eine genaue Vorstellung �ber die Anforderungen, welche die Implementierung erf�llen soll, bekommt. Anschlie�end wird Information Retrieval im Allgemeinen vorgestellt, um einen �berblick �ber die Thematik zu geben. 
Es folgt eine Erkl�rung zweier klassischer Information-Retrieval-Modelle, boolesches Retrieval und Vektorraummodell, inklusive Erl�uterung der Funktionsweisen. Anschlie�end wird beschrieben, wie sich solche Modelle in Hinsicht auf Qualit�t bewerten und vergleichen lassen.\\
Nach dem Vermitteln der notwendigen theoretischen Kenntnisse beschreibt der Implementierungsteil, auf welche Weise und in welchen Bereichen die beiden Verfahren f�r die Implementierung zum Einsatz kamen und inwieweit eine Modifizierung zur Anpassung auf die vorliegende Problemstellung erfolgte. 
Neben der internen Funktionsweise wird auch die Benutzung der Oberfl�che erl�utert, um den Anwender mit der Bedienung des Systems vertraut zu machen.  \\
Es folgt eine Zusammenfassung der Arbeit mit abschlie�ender Bewertung der Ergebnisse sowie ein Ausblick auf zuk�nftige Verbesserungsm�glichkeiten.
\newpage
\paragraph*{}
This paper discusses the conception and realization of an information retrieval system that searches efficiently in an archive of semistructured documents containing free text and metadata. The  search criteria for this system are determined by the user and can be freely connected with boolean operators. The aim of the implementation is to retrieve documents satisfying the user's information need and to present the results clearly.\\
First of all the problem is discussed in detail to give the reader a precise idea of the requirements which the implementation has to comply with. Afterwards, information retrieval in general is presented to give an overview on the topic.
The chapter about information retrieval is followed by an explanation of two classical information retrieval models called boolean retrieval and vectorspace model, including their functionality. Afterwards, it is discussed how the quality of such systems can be estimated and compared. \\
After having provided the necessary theory, the paper continues with the implementation part which explains where and how the models were used and if they have been modified to fit the given problem. 
Aside from the intern functionality, the reader learns about how to operate the system via its user interface. \\
The paper concludes with a summary, including a final result evaluation and a presentation of future prospects for system improvements.




