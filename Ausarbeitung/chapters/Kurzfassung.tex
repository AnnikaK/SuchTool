\kurzfassung

%% deutsch
\paragraph*{}
%In der Kurzfassung soll in kurzer und pr�gnanter Weise der wesentliche Inhalt der Arbeit beschrieben werden. Dazu z�hlen vor allem eine kurze Aufgabenbeschreibung, der L�sungsansatz sowie die wesentlichen Ergebnisse der Arbeit. Ein h�ufiger Fehler f�r die Kurzfassung ist, dass lediglich die Aufgabenbeschreibung (d.h. das Problem) in Kurzform vorgelegt wird. Die Kurzfassung soll aber die gesamte Arbeit widerspiegeln. Deshalb sind vor allem die erzielten Ergebnisse darzustellen. Die Kurzfassung soll etwa eine halbe bis ganze DIN-A4-Seite umfassen.

%Hinweis: Schreiben Sie die Kurzfassung am Ende der Arbeit, denn eventuell ist Ihnen beim Schreiben erst vollends klar geworden, was das Wesentliche der Arbeit ist bzw. welche Schwerpunkte Sie bei der Arbeit gesetzt haben. Andernfalls laufen Sie Gefahr, dass die Kurzfassung nicht zum Rest der Arbeit passt.



Diese Arbeit befasst sich mit der Frage, auf welche Weise ein Information Retrieval System realsiert werden kann, welches ein Archiv mit semistrukturierten Dokumenten, die sowohl Freitext als auch Metadaten enthalten, effizient nach vom Nutzer festgelegten und logisch verkn�pfbaren Kriterien durchsucht. Ziel ist es, dem Nutzer nach Abschluss des Suchvorgangs relevante Informationen zur�ckzuleifern und diese auf eine �bersichtliche Weise zu pr�sentieren. \\
Hierzu wird zun�chst die Problemstellung im Detail erl�utert. Anschlie�end wird Information Retrieval im Allgemeinen vorgestellt, um dem dem Leser einen �berblick �ber die Thematik zu verschaffen. 
Es folgt eine Vorstellung der beiden klassischen Information Retrieval Modelle boolesches Retrieval und Vektorraummodell inklusive Erl�uterung der Funktionsweise. Anschlie�end wird beschrieben, wie sich solche Modelle bewerten und vergleichen lassen.\\
Das Implementierungskapitel beschreibt, auf welche Weise und in welchen Bereichen die beiden Verfahren f�r die Implementierung dieser Arbeit zum Einsatz kamen und inwieweit eine Modifizierung zur Anpassung auf die vorliegende Problemstellung erfolgte.
Neben der internen Funktionsweise wird in dieser Ausarbeitung auch die Benutzung der Oberfl�che erl�utert, um den Anwender mit der Bedienung des Systems vertraut zu machen.
Es folgt eine Zusammenfassung der Arbeit mit abschlie�ender Bewertung der Ergebnisse sowie ein Ausblick auf weitere M�gliche Verbesserungen.

\paragraph*{}
The same in english.
