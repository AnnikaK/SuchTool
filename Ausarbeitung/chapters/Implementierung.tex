\chapter{Implementierung}
In diesem Kapitel wird beschrieben, auf welche Weise das Information Retrieval System dieser Arbeit realisiert wurde.

\section{Teilweise strukturierte Dokumente}
Besonderheit der Problemstellung ist das Vorliegen der Dokumente in semistrukturierter Form (siehe \ref{Problemstellung}. \\
 Dies bedeutet, dass zwei unterschiedliche Teilprobleme zu l�sen sind: Zum einen die Keywordsuche, welche sich auf die Suche in strukturierten Metadaten bezieht und zum anderen die Freitextsuche. Es liegt nahe, beides getrennt zu l�sen, da die Suchen unterschiedliche Anforderungen besitzen.\\
 Bevor erkl�rt wird, wie die beiden Verfahren jeweils realisiert wurden, ist es wichtig, zun�chst eine Vorstellung zu haben, wie die zu durchsuchenden Dokumente des Archivs beschaffen sind, weshalb Abbildung \ref{example} ein Beispiel zeigt.
 \begin{lstlisting}[language=lisp, caption={Beispieldokument}
 \label{example},
	language=lisp]
 
 
 (absender ("<MaxMuster@muster-mail.de>"))
 (Betreff (" Umfrage"))
 (datum ("Wed, 22 Jun 2017 07:47:51 +0200"))
 (anzahlAnhaenge 0)
 (Termin nil)
 (ABSENDER "Max_Muster")
 
 (ABSENDER-MAIL-ADRESSE "MaxMuster@muster-mail.de")
 (EMPFAENGER ("doe>> John Doe"))
 (EMPFAENGER-MAIL-ADRESSEN ("johnd@muster-mail.de"))
 (BETREFF "Umfrage")
 (EMAIL-TYP "sent")
 (QUELLBOXART "SENT")
 
 Hallo John,
 
 ich werde dir die Umfragenformulare schnellstm�glich per Post 
 zukommen lassen.
 

 Viele Gr��e
 Max Muster
 
 \end{lstlisting}
 
 In der Praxis enthalten die Dokumente oft weitaus mehr Keywords, deren Inhalt sich auch �ber mehrere Zeilen erstrecken kann, sowie Kommentare, welche vom System als Freitext interpretiert werden. \\
 In diesem Beispiel handelt es sich zwar um eine E-Mail, die Keywords k�nnen jedoch inhaltlich vollkommen unterschiedlich ausfallen. Allen Dokumenten gemeinsam ist die 
 grundlegende Struktur aus \ref{struct}.
 
 \begin{lstlisting}[language=lisp, caption=Dokumentstruktur,label={struct}]
 	(Keywordname_1 Inhalt)
 	....
 	(Keywordname_n Inhalt)
 	
 	Freitext
 		
 \end{lstlisting}
 
 
 
 \section{Keywordsuche}
 
 \section{Freitextsuche}