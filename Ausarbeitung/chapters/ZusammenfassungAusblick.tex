\chapter{Zusammenfassung und Ausblick}

%In diesem Kapitel soll die Arbeit noch einmal kurz zusammengefasst werden. Insbesondere sollen die wesentlichen Ergebnisse Ihrer Arbeit herausgehoben werden. Erfahrungen, die z.B. Benutzer mit der Mensch-Maschine-Schnittstelle gemacht haben oder Ergebnisse von Leistungsmessungen sollen an dieser Stelle pr�sentiert werden. Sie k�nnen in diesem Kapitel auch die Ergebnisse oder das Arbeitsumfeld Ihrer Arbeit kritisch bewerten. W�nschenswerte Erweiterungen sollen als Hinweise auf weiterf�hrende Arbeiten erw�hnt werden.


\section{Zusammenfassung}

Im Kapitel Einleitung und Problemstellung wurde die Aufgabenstellung erl�utert, wobei sich bereits zeigte dass diese aufgrund des semistrukturierten Aufbaus der zu durchsuchenden Dokumente eine sehr spezifische L�sung erfordern w�rde. Die Unterteilung in Metadaten und Freitext sowie das logische Verkn�pfen der Suchbedingungen stellen die wesentlichen Herausforderungen dar.
Anschlie�end wurden die notwendigen theoretischen Kenntnisse vermittelt. Die Bedeutung des Begriffs Information Retrieval wurde erkl�rt und es zeigte sich, dass diese sehr weit gefasst ist,  weshalb diese Arbeit die ausgesprochen umfangreiche Thematik nur ansatzweise behandeln kann. Es wurden die beiden bekanntesten klassichen Information Retrieval  Verfahren vorgestellt; das boolesche Retrieval und das Vektorraummodell, sowie Methoden, anhand derer sich solche Modelle im Hinblick auf Qualit�t bewerten und Vergleichen lassen. Hierbei zeigte sich, dass eine solche Bewertung im Rahmen dieser Arbeit aufgrund des Aufwands leider nicht durchf�hrbar ist. \\
Im Implementierungskapitel wurde deutlich, dass die Verwendung eines der beiden klassischen Verfahren nicht ausreicht, die Kombination aus booleschen Retrieval und Vektorraummodell jedoch hervorragend passt. Das boolesche Retrieval musste allerdings geringf�gig modifiziert werden, indem nicht nur das Vorkommen der Attribute in den Dokumenten vermerkt wird, sondern auch deren Inhalt inklusive Datentyp. Beim Vektorraummodell zeigte sich, dass sich Speicher sparen l�sst, indem alle Elemente des Vektors, welche eine 0 enthalten, nicht eingetragen werden. \\
Zuletzt zeigte sich, dass die Benutzeroberfl�che eine nicht zu untersch�tzende Herausforderung darstellt,
da der Anwender sein Informationsbed�rfnis auf m�glichst unkomplizierte Art ausdr�cken k�nnen soll. Ein Information Retrieval System muss vom Nutzer auch verstanden werden. Es wurde sich daf�r entschieden, m�glichst einfache UI-Elemente wie CheckBoxes zu w�hlen sowie die Oberfl�che in drei thematische Bereiche zu gliedern, um die notwendige �bersicht zu bieten.




 \section{Ausblick: W�nschenswerte Erweiterungen}
Einige Erweiterungen konnten aus zeitlichen Gr�nden in der Implementierung nicht umgesetzt werden, sind jedoch w�nschenswert. \\
Die Im Kapitel Grundbegriffe beschriebene Lemmatisierung (siehe Abschnitt \ref{lemmatisierung}) w�rde eine gro�e Verbesserung hinsichtlich des Speicherbedarfs bedeuten, da weniger Terme gespeichert werden m�ssen. Zudem w�rde das Erg�nzen um Lemmatisierung auch die Suche verbessern, da auch Dokumente mit zum Suchbegriff verwandten W�rtern als Treffer gewertet werden. \\
Weiterhin w�nschenswert ist die Erg�nzung um Spelling Correction, d.h. um eine Rechtschreibkorrektur die �hnliche Suchbegriffe vorschl�gt, falls der Anwender sich bei der Eingabe vertippt hat. Dies w�rde die Anwenderfreundlichkeit des Systems erheblich steigern, insbesondere in F�llen in welchen der Anwender seinen Tippfehler nicht bemerkt und unerwartet keine Resultate erh�lt. Da Nutzer in der Regel Spelling Correction aus der allt�glichen Websuche gew�hnt sind, ist es w�nschenswert auch dieses Information Retrieval System damit auszustatten.


%-Stopwords \\
%-Lemmatisierung \\
%-Spelling Correction \\
