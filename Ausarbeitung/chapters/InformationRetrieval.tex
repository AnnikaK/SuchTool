\chapter{Information Retrieval}
Dieses Kapitel soll dem Leser einen �berblick dar�ber vermitteln, wof�r der Begriff \glqq Information Retrieval \grqq steht.

\section{Bedeutung}
Der englische Begriff \glqq Information Retrieval \grqq{} l�sst sich mit \glqq Informationsr�ckgewinnung\grqq{} ins Deutsche �bersetzen. Hierbei wird explizit von R�ckgewinnung gesprochen, da ein Information Retrieval System keine neuen Informationen erzeugt.\\
 Hier stellt sich sofort die Frage nach der tats�chlichen Aufgabe eines solchen Systems. Bevor dies beschrieben wird, soll zun�chst gekl�rt werden, was mit dem Teilbegriff \glqq Information\grqq{} gemeint ist.

%hier referenz angeben

\subsection{Information}
Die Bedeutung des Begriffsbestandteils \qlqq Information \grqq erweist sich als weitaus schwieriger darzustellen: Was ist eine Information?\\
 Eine einheitliche Definition hierzu l�sst sich nicht finden, zumal es zahlreiche Betrachtungsweisen aus unterschiedlichen Disziplinen gibt. F�r diese Arbeit ist lediglich die Perspektive der Disziplin Informatik relevant, darum wurde hieraus ein Definitionsansatz gew�hlt.
 
 
\subsubsection{Information}

\section{Information Retrieval}


\begin{definition} (Information Retrieval) \\
Mit Information Retrieval, kurz IR, wird das Auffinden von in unstrukturierter Form vorliegender und ein Informationsbed�rfnis befriedigender Materialien innerhalb gro�er Sammlungen bezeichnet. \\
\end{definition}
Mit unstrukturierten Materialien sind hierbei meist Dokumente in Textform gemeint. �blicherweise liegen die Sammlungen auf dem Computer gespeichert vor (\cite{Manning:08}, S.1).