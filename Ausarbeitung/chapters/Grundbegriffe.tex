\chapter{Grundbegriffe}
Unabh�ngig vom jeweiligen Modell gibt es einige Grundbegriffe, welche im Zusammenhang mit Information-Retrieval-Verfahren immer wieder auftauchen und die darum vorab vorgestellt werden.

\section{Dokument}
 Wenn von Dokumenten gesprochen wird, ist hiermit die Einheit gemeint, auf der das Retrieval stattfindet. Findet es beispielsweise auf einem Buch statt, stellen Kapitel eine m�gliche Einheit dar. Andererseits kann es f�r den Nutzer unbefriedigend sein, ganze Kapitel als Ergebnis zu erhalten, weshalb einzelne Seiten eine weitere m�gliche, feinere Einheit sind. Selbst kurze Textmemos sind g�ltige Einheiten, d.h. die Gr��e der Dokumente unterliegt keiner Beschr�nkung (\cite{Manning:08}, S.4). \\
 Im Falle der gegebenen Problemstellung (siehe Abschnitt \ref{Problemstellung}) entspricht ein Dokument einem semistrukturierten Textdokument des zu durchsuchenden Archivs, wobei dessen L�nge stark variieren kann. Bei Einschr�nkung auf den Anwendungsfall E-Mail-Archiv kann ein Dokument auch als E-Mail betrachtet werden.

\section{Anfrage}
Eine Anfrage (engl. \textit{query}) wird vom Anwender in den Computer eingegeben, um sein Informationsbed�rfnis zu befriedigen (\cite{Manning:08}, S.5). 
Anfragen k�nnen je nach Information-Retrieval-System vollkommen unterschiedlich strukturiert sein. Wichtig ist, dass der Nutzer wei�, wie er seine Anfrage syntaktisch korrekt stellen muss, um mehr �ber das gew�nschte Thema zu erfahren, was insbesondere bei booleschen Information-Retrieval-Systemen (siehe Kapitel \ref{bool}) komplex werden kann.


\section{Indexierung}
Damit Dokumente eines Archivs von Information-Retrieval-Systemen verarbeitet werden k�nnen, m�ssen diese mit einem eindeutigen Index, der \textit{docID} (kurz f�r \textit{document identification}), versehen werden, sodass schnell darauf zugegriffen werden kann. Bei der \textit{docID} handelt es sich meist um einen ganzzahligen Wert (\cite{Manning:08}, S.7). Zudem m�ssen die Dokumentinhalte indexiert werden, wozu die Texte in einzelne, mit einem eindeutigen Index versehene Einheiten zerlegt werden, sodass auch hierauf effiziente Zugriffe erfolgen k�nnen.
Dieses Vorgehen wird als Indexierung bezeichnet und ist unabdingbar, da ansonsten f�r jede Anfrage erneut �ber die gesamten Dokumentinhalte iteriert werden m�sste, was ineffizient und f�r den Nutzer unzumutbar langsam w�re (\cite{Manning:08}, S.3). 


\subsection{Term und Vokabular}
\label{lemmatisierung}
Die indexierten Einheiten, in welche die Dokumente zerlegt werden, sind unter dem Begriff Terme bekannt (\cite{Manning:08}, S.3).
Terme sind im h�ufigsten und einfachsten Fall W�rter eines Textes, dies muss jedoch nicht zwangsl�ufig zutreffen.\\
Manche Systeme reduzieren W�rter beispielsweise auf deren Stammformen, um �hnliche W�rter zu einem einzigen Term zusammenzufassen. Alternativ lassen sie sich neben dem Wortstamm auch auf ihre grammatikalische Grundform reduzieren. Die Reduktion der W�rter auf Wortstamm bzw. Grundform wird als Lemmatisierung oder Stemming bezeichnet. Auf diese Weise m�ssen weniger Terme verwaltet werden, was den Speicherbedarf reduziert. Au�erdem werden mehr relevante Dokumente gefunden, da auch zum Suchbegriff verwandte W�rter zu einem Treffer f�hren (\cite{Ferber:03}, S.40-41). \\
 Die Menge aller Terme eines Archivs wird als Vokabular bezeichnet (\cite{Manning:08}, S.6). \\




\subsection{Stoppw�rter}
\label{stop}
Nicht jedes Wort wird bei der Indexierung zu einem Term verarbeitet: Handelt es sich um sehr h�ufig auftretende und zum Sinn des Textes wenig beitragende W�rter, wie z.B. \glqq und\grqq{} oder \glqq dann\grqq{}, k�nnen diese wegfallen, um Speicherplatz zu sparen (\cite{Ferber:03}, S.37). Au�erdem wird durch das Ignorieren unwichtiger Terme die Suche erheblich beschleunigt. Wie diese Beschleunigung genau zustande kommt, h�ngt vom jeweiligen Information-Retrieval-Verfahren ab.\\



