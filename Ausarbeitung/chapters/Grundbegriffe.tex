\chapter{Grundbegriffe}
Unabh�ngig vom jeweiligen Modell gibt es einige Grundbegriffe, welche im Zusammenhang mit Information Retrieval Verfahren immer wieder auftauchen und die darum vorab vorgestellt werden sollen. 

\section{Anfrage}
Eine Anfrage (engl. query) ist das, was der Anwender in den Computer eingibt, um sein Informationsbed�rfnis zu befriedigen (\cite{Manning:08}, S.5). \\
Anfragen k�nnen je nach Information Retrieval System vollkommen unterschiedlich strukturiert sein. Wichtig ist, dass der Nutzer wei� wie er seine Anfrage stellen muss, um mehr �ber das gew�nschte Thema zu erfahren, was bei booleschen Modellen (siehe Kapitel \ref{bool}) recht komplex werden kann. \\


\section{Indexierung}
Damit Dokumente eines Archivs von Information Retrieval Modellen verarbeitet werden k�nnen, m�ssen diese in einzelne Einheiten zerlegt werden, wobei jede Einheit mit einem eindeutigen Index versehen sein muss, um hinterher mit einem schnellen Zugriff abrufbar zu sein. \\
Zudem m�ssen auch die Dokumente selbst mit einem Index versehen werden, genannt \textit{docID} (kurz f�r \textit{document identification}), um auf deren enthaltende Einheiten schnell zugreifen zu k�nnen. Dabei handelt es sich meist um einen ganzzahligen Wert (\cite{Manning:08}, S.7).\\
Dieses Vorgehen wird als Indexierung bezeichnet und ist unabdingbar, da ansonsten f�r jede Anfrage erneut �ber die gesamten Dokumentinhalte iteriert werden m�sste, was ineffizient und f�r den Nutzer unzumutbar langsam w�re (\cite{Manning:08}, S.3). \\

\subsection{Term und Vokabular}
Die indizierten Einheiten, in welche die Dokumente zerlegt werden, sind unter dem Begriff Term bekannt (\cite{Manning:08}, S.3).\\
Terme sind im h�ufigsten Fall einfach W�rter eines Textes, dies muss jedoch nicht zwangsl�ufig der Fall sein.\\
Manche Systeme reduzieren W�rter beispielsweise auf deren Stammformen, um �hnliche W�rter zu einem einzigen Term zusammenzufassen. Alternativ lassen sich W�rter neben dem Wortstamm auch auf ihre grammatikalische Grundform reduzieren. Das Reduzieren der W�rter wird allgemein als Lemmatisierung oder Stemming bezeichnet.\\ Auf diese Weise m�ssen weniger Terme verwaltet werden, was den Speicherbedarf reduziert. Au�erdem k�nnen leichter �hnliche Dokumente gefunden werden, da auch zum Suchbegriff verwandte W�rter zu einem Treffer f�hren (\cite{Ferber:03}, S.40-41). \\
 Die Menge aller unterschiedlichen Terme eines Archivs wird als Vokabular bezeichnet (\cite{Manning:08}, S.6). \\




\subsection{Stopwords}
Nicht jedes Wort wird bei der Indexierung zu einem Term: Handelt es sich um sehr h�ufig auftretende und zum Sinn des Textes wenig beitragende W�rter, wie z.B. \glqq und\grqq{} oder \glqq dann\grqq{}, k�nnen diese wegfallen, um Speicherplatz zu sparen (\cite{Ferber:03}, S.37). Au�erdem kommen durch ignorieren unwichtiger Terme weniger Dokumente infrage, was die Suche beschleunigt.\\



