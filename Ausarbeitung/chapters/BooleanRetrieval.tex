\chapter{Boolesches Retrieval}
\label{bool}
Dieses Kapitel stellt das klassische Information-Retrieval-Verfahren boolesches Retrieval (engl.Boolean Retrieval) vor. \\

\section{Eigenschaften des Verfahrens}
Boolesches Retrieval �berpr�ft Dokumente darauf, ob eine bestimmte Bedingung zutrifft.\\
Somit gibt es nur die Unterteilung in passende Dokumente und solche, welche diese Bedingung nicht erf�llen. Eine dar�ber hinausgehende Bewertung der Ergebnisse findet nicht statt, was zu einer ungeordneten Ergebnismenge f�hrt (\cite{Ferber:03}, S.33). Das fehlende Ranking der Ergebnisse ist ein h�ufiger Kritikpunkt des Verfahrens.\\
 
 
\section{Funktionsprinzip}
Boolesches Retrieval basiert auf Mengenoperationen. Deshalb werden Dokumente Mengen zugeordnet, die jeweils durch bestimmte Attribute charakterisiert sind. \\
Dokument bezeichnet die Einheit, auf welcher das Retrieval stattfindet. Ein Dokument kann deshalb eine kleine Textmemo, aber auch ein ganzes Buchkapitel sein (\cite{Manning:08}, S.4).\\
\subsection{Attribut}
Ein solches Attribut ist eine Abbildung, welche jedem Dokument einen Wert f�r dieses Attribut zuordnet. Die Abbildung erzeugt somit Attribut-Wert-Paare, was in Formel \ref{attr} gezeigt wird. 

\begin{equation}
\label{attr}
t: D \rightarrow T, t(d) = t_i
\end{equation}


Hierbei bezeichnet $t$ die Abbildung (d.h. das Attribut), $D$ die Menge aller Dokumente und $T$ den Wertebereich des Attributs $t$. \\
 Der Attributwert $t_i$ mit $t_i \in T$ und $i \in \mathbb{N}$ wird durch die Abbildung $t$ dem Dokument $d \in D$ zugeordnet. \\

\subsection{Anfragen}
 \label{formelzeugs}
\subsubsection{Elementare boolesche Anfrage}
Ein Attribut-Wert-Paar wird auch als elementare boolesche Anfrage bezeichnet. Bei der elementaren booleschen Anfrage  $(t,t_1)$ werden zum Beispiel alle Dokumente gesucht, deren Attribut $t$ den Wert $t_1$ annimmt. \\
Die Ergebnismenge $D_t,_{t_i}$ f�r eine Anfrage $(t,t_i)$ kann demnach wie in Formel \ref{ergMen} charakterisiert werden. 

\begin{equation}
\label{ergMen}
D_t,_{t_i} =  \{d \in D | t(d) = t_i\} \\
\end{equation}

\subsubsection{Verkn�pfung}
Werden elementare Anfragen miteinander logisch verkn�pft, so werden  abh�ngig vom jeweiligen booleschen Operator bestimmte Mengenoperationen auf den Ergebnismengen der elementaren Anfragen ausgef�hrt. \\
Die m�glichen booleschen Operatoren sind hierbei $AND$, $OR$ und $NOT$.\\
$(t,t_1)$ $AND$ $(s,s_1)$ bedeutet, dass alle Dokumente gesucht sind, bei denen sowohl $t(d) = t_1$ als auch $s(d) = s_1$ gilt. Die erforderliche Mengenoperation ist deshalb der Durchschnitt aus den beiden Ergebnismengen, was in Formel \ref{intersection} gezeigt wird.

\begin{equation}
\label{intersection}
D_t,_{t_1} \cap D_s,_{s_1}
\end{equation}


Wird hingegen der Operator $OR$ verwendet, wird die Mengenoperation Vereinigung ben�tigt (siehe Formel \ref{union}), da alle Dokumente mit  $t(d) = t_1$ oder $s(d) = s_1$ gesucht sind. \\


\begin{equation}
\label{union}
D_t,_{t_1} \cup D_s,_{s_1}
\end{equation}


Au�erdem kann der un�re Operator $NOT$ verwendet werden, welcher das Komplement der Ergebnismenge erzeugt. F�r die Anfrage $NOT$ $(t,t_1)$ muss erst die Menge aller Dokumente bestimmt werden, bei denen $t(d) = t_1$ zutrifft, um diese anschlie�end von der Gesamtmenge aller Dokumente abzuziehen. Dies wird in Formel \ref{minus} dargestellt.


\begin{equation}
\label{minus}
D \setminus D_t,_{t_1}
\end{equation}



Da bei jeder Mengenoperation als Ergebnis neue Mengen entstehen, lassen sich hierauf erneut die oben beschriebenen Operatoren anwenden. Auf diese Weise k�nnen Anfragen beliebig tief geschachtelt werden (\cite{Ferber:03}, S.34).









