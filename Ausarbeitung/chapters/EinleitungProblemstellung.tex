\chapter{Einleitung und Problemstellung}

\section{Einleitung}


Tag f�r Tag treffen neue E-Mail Nachrichten mit den unterschiedlichsten Inhalten im Posteingang ein, weshalb die Anzahl gesendeter und empfangener Mails schnell un�bersichtlich gro� wird. \\
 Will der Nutzer eine Weile sp�ter auf eine bestimmte Nachricht erneut zugreifen, steht er oftmals vor der Problematik, mit der manuellen Suche aufgrund der schieren Datenmenge �berfordert zu sein. \\
 Besonders schwierig wird es f�r ihn, wenn er nach bestimmten Inhalten sucht, sich jedoch weder an das genaue Datum noch an den Absender erinnert. An dieser Stelle hilft dem Nutzer ein Suchprogramm, mit dem er Suchkriterien beliebig logisch verkn�pfen kann. \\
 Wei� er beispielsweise noch, dass die Nachricht im Mai ankam und eine wichtige Adresse beinhaltet, k�nnte seine Suchanfrage \glqq Datum = Mai und Freitext = Adresse \grqq lauten.\\
  Das Suchprogramm pr�sentiert dann eine Liste f�r ihn in Frage kommender Dokumente, was die Suche auf einen kleinen Kreis relevanter Treffer einschr�nkt. \\
  Die Suche nach bestimmten 
 
 


Begonnen werden soll mit einer Einleitung zum Thema, also Hintergrund und Ziel erl�utert werden.

Weiterhin wird das vorliegende Problem diskutiert: Was ist zu l�sen, warum ist es wichtig, dass man dieses Problem l�st und welche L�sungsans�tze gibt es bereits. Der Bezug auf vorhandene oder eben bisher fehlende L�sungen begr�ndet auch die Intention und Bedeutung dieser Arbeit. Dies k�nnen allgemeine Gesichtspunkte sein: Man liefert einen Beitrag f�r ein generelles Problem oder man hat eine spezielle Systemumgebung oder ein spezielles Produkt (z.B. in einem Unternehmen), woraus sich dieses noch zu l�sende Problem ergibt.

Im weiteren Verlauf wird die Problemstellung konkret dargestellt: Was ist spezifisch zu l�sen? Welche Randbedingungen sind gegeben und was ist die Zielsetzung? Letztere soll das
beschreiben, was man mit dieser Arbeit (mindestens) erreichen m�chte.


\section{Problemstellung}

Teilweise strukturiert, teilweise Freitext, man will zeugs finden -> relativ allgemein halten


\section{Teilprobleme}
\subsection{Unterscheidung zwischen Metadaten und Freitext}
\subsection{Metadatensuche}
\subsection{Freitextsuche}
\subsection{Verkn�pfung mit UND/ODER}
\subsection{Benutzeroberfl�che}