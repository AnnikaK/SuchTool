%%%%%%%%%%%%%%%%%%% vorlage.tex %%%%%%%%%%%%%%%%%%%%%%%%%%%%%
%
% LaTeX-Vorlage zur Erstellung von Projekt-Dokumentationen
% im Fachbereich Informatik der Hochschule Trier
%
% Basis: Vorlage svmono des Springer Verlags
%
%%%%%%%%%%%%%%%%%%%%%%%%%%%%%%%%%%%%%%%%%%%%%%%%%%%%%%%%%%%%%

\documentclass[envcountsame,envcountchap, deutsch]{i-studis}

\usepackage{makeidx}         	% Index
\usepackage{multicol}        	% Zweispaltiger Index
%\usepackage[bottom]{footmisc}	% Erzeugung von Fu�noten

%%-----------------------------------------------------
%\newif\ifpdf
%\ifx\pdfoutput\undefined
%\pdffalse
%\else
%\pdfoutput=1
%\pdftrue
%\fi
%%--------------------------------------------------------
%\ifpdf
\usepackage[pdftex]{graphicx}
\usepackage{epstopdf}
\usepackage[pdftex,plainpages=false]{hyperref}
%\else
%\usepackage{graphicx}
%\usepackage[plainpages=false]{hyperref}
%\fi

%%-----------------------------------------------------
\usepackage{color}				% Farbverwaltung
%\usepackage{ngerman} 			% Neue deutsche Rechtsschreibung
\usepackage[english, ngerman]{babel}
\usepackage[latin1]{inputenc} 	% Erm�glicht Umlaute-Darstellung
%\usepackage[utf8]{inputenc}  	% Erm�glicht Umlaute-Darstellung unter Linux (je nach verwendetem Format)

%-----------------------------------------------------
\usepackage{listings} 			% Code-Darstellung
\lstset
{
	basicstyle=\scriptsize, 	% print whole listing small
	keywordstyle=\color{blue}\bfseries,
								% underlined bold black keywords
	identifierstyle=, 			% nothing happens
	commentstyle=\color{red}, 	% white comments
	stringstyle=\ttfamily, 		% typewriter type for strings
	showstringspaces=false, 	% no special string spaces
	framexleftmargin=7mm, 
	tabsize=3,
	showtabs=false,
	frame=single, 
	rulesepcolor=\color{blue},
	numbers=left,
	linewidth=146mm,
	xleftmargin=8mm
}
\usepackage{textcomp} 			% Celsius-Darstellung
\usepackage{amssymb,amsfonts,amstext,amsmath}	% Mathematische Symbole
\usepackage[german, ruled, vlined]{algorithm2e}
\usepackage[a4paper]{geometry} % Andere Formatierung
\usepackage{bibgerm}
\usepackage{array}
\hyphenation{Ele-men-tar-ob-jek-te  ab-ge-tas-tet Aus-wer-tung House-holder-Matrix Le-ast-Squa-res-Al-go-ri-th-men} 		% Weitere Silbentrennung bei Bedarf angeben
\setlength{\textheight}{1.1\textheight}
\pagestyle{myheadings} 			% Erzeugt selbstdefinierte Kopfzeile
\makeindex 						% Index-Erstellung


%--------------------------------------------------------------------------
\begin{document}
%------------------------- Titelblatt -------------------------------------
\title{Konzeption und Realisierung eines Systems zur Informationssuche in einem Dokumentenarchiv basierend auf Textinhalt und Metadaten}
\subtitle{Conception and Realization of a Information Search System in a Document Archive based on Text Content and Meta Data}
%---- Die Art der Dokumentation kann hier ausgew�hlt werden---------------
%\project{Bachelor-Projektarbeit}
\project{Bachelor-Abschlussarbeit}
%\project{Master-Projektstudium}
%\project{Master-Abschlussarbeit}
%\project{Seminar zur Vorlesung ...}
%\project{Hausarbeit zur Vorlesung ...}
%--------------------------------------------------------------------------
\supervisor{Professor Karl Hans Bl\"asius} 		% Betreuer der Arbeit
\author{Annika Kremer} 							% Autor der Arbeit
\address{Trier,} 							% Im Zusammenhang mit dem Datum wird hinter dem Ort ein Komma angegeben
\submitdate{Abgabedatum} 				% Abgabedatum
%\begingroup
%  \renewcommand{\thepage}{title}
%  \mytitlepage
%  \newpage
%\endgroup
\begingroup
  \renewcommand{\thepage}{Titel}
  \mytitlepage
  \newpage
\endgroup
%--------------------------------------------------------------------------
\frontmatter 
%--------------------------------------------------------------------------
\preface

Ein Vorwort ist nicht unbedingt n�tig. Falls Sie ein Vorwort schreiben, so ist dies der Platz, um z.B. die Firma vorzustellen, in der diese Arbeit entstanden ist, oder einigen Leuten zu danken, die in irgendeiner Form positiv zur Entstehung dieser Arbeit beigetragen haben. Auf keinen Fall sollten Sie im Vorwort die Aufgabenstellung n�her erl�utern oder vertieft auf technische Sachverhalte eingehen.				% Vorwort (optional)
\kurzfassung

%% deutsch
\paragraph*{}
%In der Kurzfassung soll in kurzer und pr�gnanter Weise der wesentliche Inhalt der Arbeit beschrieben werden. Dazu z�hlen vor allem eine kurze Aufgabenbeschreibung, der L�sungsansatz sowie die wesentlichen Ergebnisse der Arbeit. Ein h�ufiger Fehler f�r die Kurzfassung ist, dass lediglich die Aufgabenbeschreibung (d.h. das Problem) in Kurzform vorgelegt wird. Die Kurzfassung soll aber die gesamte Arbeit widerspiegeln. Deshalb sind vor allem die erzielten Ergebnisse darzustellen. Die Kurzfassung soll etwa eine halbe bis ganze DIN-A4-Seite umfassen.

%Hinweis: Schreiben Sie die Kurzfassung am Ende der Arbeit, denn eventuell ist Ihnen beim Schreiben erst vollends klar geworden, was das Wesentliche der Arbeit ist bzw. welche Schwerpunkte Sie bei der Arbeit gesetzt haben. Andernfalls laufen Sie Gefahr, dass die Kurzfassung nicht zum Rest der Arbeit passt.



Diese Arbeit befasst sich mit der Konzeption und Realisierung eines Information-Retrieval-Systems, welches ein Archiv mit semistrukturierten Dokumenten, die sowohl Freitext als auch Metadaten enthalten, effizient nach vom Nutzer festgelegten und logisch verkn�pfbaren Kriterien durchsucht. Ziel der Implementierung ist es, dem Nutzer nach Abschluss des Suchvorgangs zu seinem Informationsbed�rfnis passende Dokumente zur�ckzuliefern und diese auf eine �bersichtliche Weise zu pr�sentieren. \\
Zun�chst wird die Problemstellung im Detail erl�utert, damit der Leser eine genaue Vorstellung �ber die Anforderungen, welche die Implementierung erf�llen soll, bekommt. Anschlie�end wird Information Retrieval im Allgemeinen vorgestellt, um einen �berblick �ber die Thematik zu geben. 
Es folgt eine Vorstellung der beiden klassischen Information Retrieval Modelle boolesches Retrieval und Vektorraummodell inklusive Erl�uterung der Funktionsweisen. Anschlie�end wird beschrieben, wie sich solche Modelle in Hinsicht auf deren Qualit�t bewerten und vergleichen lassen.\\
Nach dem Vermitteln der notwendigen theoretischen Kenntnisse beschreibt der Implementierungsteil, auf welche Weise und in welchen Bereichen die beiden Verfahren f�r die Implementierung zum Einsatz kamen und inwieweit eine Modifizierung zur Anpassung auf die vorliegende Problemstellung erfolgte. 
Neben der internen Funktionsweise wird auch die Benutzung der Oberfl�che erl�utert, um den Anwender mit der Bedienung des Systems vertraut zu machen.  \\
Es folgt eine Zusammenfassung der Arbeit mit abschlie�ender Bewertung der Ergebnisse sowie ein Ausblick auf zuk�nftige Verbesserungsm�glichkeiten.
\newpage
\paragraph*{}
This paper discusses the conception and realization of an information retrieval system that searches efficiently in semistructured data containing free text and metadata. The  search criteria for this system are determined by the user and can be freely connected with boolean operators. The aim of the implementation is to retrieve documents satysfying the user's information need and to present the results clearly.\\
First the problem is discussed in detail to give the reader a precise idea of the requirements which the implementation has to fit. Afterwards, information retrieval in general is presented to give an overview on the topic.
The chapter about information retrieval is followed by a presentation of two classical information retrieval model including their functionality called boolean retrieval and vectorspace model. Afterwards, it is discussed how the quality of such systems can be estimated and compared. \\
After having provided the necessary theory, the paper continues with the implementation part, which explains which models were used in which way and if they have been modified in any way to fit this problem. 
Aside from the intern functionality, the reader learns about how to operate with the system via the given user interface. \\
The paper concludes with a summary including a final result evaluation and a presentation of future prospects for system improvements.




 			% Kurzfassung Deutsch/English
\tableofcontents 						% Inhaltsverzeichnis
\listoffigures 							% Abbildungsverzeichnis (optional)
\listoftables 							% Tabellenverzeichnis (optional)
%--------------------------------------------------------------------------
\mainmatter                        		% Hauptteil (ab hier arab. Seitenzahlen)
%--------------------------------------------------------------------------
% Die Kapitel werden in separaten .tex-Dateien abgelegt und hier eingebunden.
\chapter{Einleitung und Problemstellung}

\section{Einleitung}

Wer kennt es nicht: Tagt�glich treffen neue E-Mails ein, sodass das Postfach sich immer weiter f�llt. \\
Schnell ist es passiert, dass die Menge an Nachrichten un�bersichtlich gro� wird, was zu einem ernsthaften Problem wird, sobald man etwas bestimmtes darin wiederfinden m�chte. \\
Die ben�tigte Mail war wichtig, weil sie eine bestimmte Kontaktadresse enthielt, aber wie war nochmal der Absender? L�ngst vergessen. Das genaue Datum? Leider ist nur noch der Monat bekannt. Wer jetzt manuell suchen muss, ist an dieser Stelle verloren. \\
Abhilfe schafft ein Information Retrieval System, mit dem man bestimmte Suchkriterien eingeben und beliebig miteinander kombinieren kann. \\
 So bekommt der Nutzer genau die Nachrichten pr�sentiert, die sein Informationsbed�rfnis am ehesten bedienen, und  braucht sich nicht erst durch hunderte von Mails durchzuarbeiten. In diesem Fall k�nnte die Person beispielsweise im Freitext nach dem Wort \glqq Kontaktadresse\grqq{} suchen und weitere Kriterien wie \glqq Datum = Juni \grqq{} hinzuf�gen. \\
Besonderheit ist das beliebige Verkn�pfen: Der Nutzer kann sich entscheiden, ob er nur Resultate akzeptiert, auf die Beides zutrifft, oder ob es bereits reicht, wenn eines der Kriterien erf�llt ist. Dies erlaubt eine sehr individuelle, auf den Benutzer zugeschnittene Suche. \\
Ein solches System ist nicht nur f�r das Alltagsbeispiel E-Mail-Ordner, d.h. Posteingang, Postausgang etc. wertvoll, sondern l�sst sich auch auf jede andere Art von Dokumentenarchiv, dessen Dokumente sowohl Freitext als auch Metadaten beinhalten, anwenden. 





\section{Problemstellung}

Ziel der Arbeit ist die Konzeption und Realisierung eines Systems zur Informationssuche (engl. Information Retrieval System) in einem Dokumentenarchiv, wobei die Dokumente von teilweise strukturierter Natur sind. \\
Dies bedeutet, dass sie sowohl gew�hnlichen Freitext als auch Metadaten enthalten. Der Nutzer soll spezifizieren k�nnen, in welchen Metadaten er suchen m�chte, zudem soll die Freitextsuche ausw�hlbar sein. Alle Suchanfragen sollen hierbei beliebig mit den booleschen Operatoren \textit{AND} (engl. und) sowie \textit{OR} (engl. oder) verkn�pfbar sein. \\
Hauptanwendungszweck des Systems sind E-Mail-Archive wie Posteingang und Postausgang, allerdings soll das System so flexibel sein, dass es auch auf andere teilweise strukturierte Dokumentenarchive anwendbar ist. 


\section{Teilprobleme}
Aus der Aufgabenstellung ergeben sich die im folgenden beschriebenen Teilprobleme.


\subsection{Dynamisches einlesen der Metadaten}
Die genauen Metadaten sind, da das System flexibel sein soll, vor dem Ausf�hren des Systems noch nicht bekannt. Demnach muss das IR-System die Namen der Metadaten beim Starten des Programms dynamisch einlesen und diese dem Nutzer auf der grafischen Oberfl�che anzeigen.

\subsection{Unterscheidung Metadaten und Freitext}
Damit das System die Metadaten dynamisch einlesen kann, m�ssen die folgenden Punkte erf�llt sein:
\begin{enumerate}
	\item Das System muss zwischen Metadaten und Freitext unterscheiden k�nnen.
	\item Metadaten setzen sich aus Name und Inhalt zusammen, weshalb beides erkannt und voneinander abgegrenzt werden muss. Im folgenden werden die Namen als \glqq Keywords \grqq{} bezeichnet.
	\item Der Inhalt kann unterschiedlichen Datentyps sein, z.B. String oder Liste, weshalb dieser bestimmt werden muss.
\end{enumerate}


\subsection{Metadatensuche}
Es muss erkannt werden, welche Metadaten der Nutzer ausgew�hlt hat und in genau diesen muss, unter Ber�cksichtigung des jeweiligen Datentyps der Inhalte, gesucht werden. \\
Im Gegensatz zur Freitextsuche muss zu jedem Dokument vor der Suche zun�chst gepr�ft werden, ob das entsprechende Schl�sselwort �berhaupt darin auftritt.

\subsection{Freitextsuche}
Bei der Freitextsuche ist die Wortzahl weitaus gr��er als bei der Metadatensuche. Daraus resultieren zwei Probleme:
\begin {enumerate}
\item Wie kann effizient in gro�en Wortmengen gesucht werden?
\item Wie kann die Suche bei begrenztem Speicher bew�ltigt werden?
\end{enumerate}

Zudem stellt sich die Frage nach einem geeigneten Verfahren, das bei l�ngeren Anfragen auch teilweise passende Ergebnisse liefert und messen kann, wie gut die erzielten Treffer zur Nutzeranfrage passen.


\subsection{Verkn�pfung mit UND/ODER}
Alle Anfragen sollen beliebig mit den logischen Operatoren \textit{AND}, \textit{OR} sowie \textit{NOT} verkn�pfbar sein. Dies beinhaltet die folgenden Problemstellungen:
\begin{enumerate}
\item Keywordsuche und Freitextsuche m�ssen miteinander verkn�pft werden.
\item Sind meherere Keywords ausgew�hlt, m�ssen die Teilergebnisse verkn�pft werden.
\item Stellt der Nutzer mehrere Anfragen, m�ssen die Ergebnisse der einzelnen Anfragen verkn�pft werden.
	\end{enumerate}






\subsection{Benutzeroberfl�che}
Der Nutzer ben�tigt eine verst�ndliche Benutzeroberfl�che, die es ihm erm�glicht, seine Suchanfragen beliebig zusammenzustellen. Hierzu muss die Oberfl�che folgende grundlegenden Funktionalit�ten aufweisen:
\begin{enumerate}
	\item Das Suchverzeichnis, d.h. das Dokumentenarchiv in welchem die Suche stattfindet, muss ausw�hlbar sein. 
	\item Alle im Archiv auftretenden Keywords sowie die Freitextsuche m�ssen ausw�hlbar sein.
	\item Logische Operatoren (AND,OR,NOT) zur Verkn�pfung m�ssen ausw�hlbar sein.
	\item Die Suchanfrage muss f�r den Nutzer verst�ndlich angezeigt werden.
	
	
\end{enumerate}

\chapter{Information Retrieval}
Dieses Kapitel soll dem Leser einen �berblick �ber die Bedeutung des Begriffs \glqq Information Retrieval\grqq{} vermitteln.


\section{Bedeutung}


Der aus dem Englischen stammende Begriff \glqq Information Retrieval\grqq{} l�sst sich mit \glqq Informationsr�ckgewinnung\grqq{} ins Deutsche �bersetzen (\cite{Academic:12}). Hierbei wird explizit von \textit{R�ck}gewinnung gesprochen, da keine neuen Informationen erzeugt werden, sondern auf bereits existierende zugegriffen wird. 
Bevor auf die genaue Bedeutung eingegangen wird, erfolgt zun�chst die in dieser Arbeit verwendete Definition des im Ausdruck enthaltenen Teilbegriffs \glqq Information\grqq{}. 

 
\subsection{Information}
\label{info}
Der Begriff stammt von dem lateinischen Wort \textit{informare}, was sich mit \glqq Gestalt geben\grqq{} �bersetzen l�sst und im �bertragenen Sinne \glqq jemanden durch Unterweisung bilden\grqq{} bedeutet (\cite{Claus:06}, S.314-315). \\
Dies betont den Aspekt, dass eine Information stets einen Empf�nger besitzt, welcher \glqq gebildet\grqq{} wird. Dieser kann eine Person, aber auch ein geeignetes, nach au�en wirksames System sein. 
Erst das Aufnehmen und korrekte Interpretieren durch einen Empf�nger macht aus Daten als Informationstr�gern tats�chlich Informationen. 
Die Informationen m�ssen deshalb auf eine von Menschen bzw. Systemen interpretierbare Weise dargestellt werden, beispielsweise durch alphabetische Zeichen. Zudem muss es hierf�r einen geeigneten Tr�ger, z.B. ein Textdokument, geben (\cite{Claus:06}, S.314-315). \\
Informationen lassen sich in die folgenden drei Bestandteile zerlegen (\cite{Claus:06}, S.314-315): 

\begin{itemize}
	\item \textit{Syntaktischer Teil:} Ist die Struktur der Information syntaktisch zul�ssig? Beispiel hierf�r ist die Einhaltung von Rechtschreibung und Grammatik bei Texten.
	\item \textit{Semantischer Teil:} Welche inhaltliche Bedeutung besitzt die Information? 
	\item \textit{Pragmatischer Teil:} Welchem Zweck dient sie?
\end{itemize} 


\subsection{Begriffsdefinition}
Nach dem der Teilbegriff \glqq Information\grqq{} vorgestellt wurde, wird in diesem Abschnitt auf die Bedeutung von Information Retrieval eingegangen.
Auch hier ist es problematisch, eine einheitliche Definition zu finden. Eine m�gliche Erkl�rung lautet wie folgt: \\


\begin{definition} (Information Retrieval) \\
	\label{Def}
Mit Information Retrieval, kurz IR, wird das Auffinden von in unstrukturierter Form vorliegender und ein Informationsbed�rfnis befriedigender Materialien innerhalb gro�er Sammlungen bezeichnet (\cite{Manning:08}, S.1). \\


\end{definition}
Mit unstrukturierten Materialien sind hierbei meist Dokumente in Textform gemeint, \glqq unstrukturiert\grqq{} bezeichnet jedoch allgemein alle Dokumente ohne eine klar vorgegebene, f�r einen Computer leicht zu verarbeitende Struktur. In der Praxis sind nur die wenigsten Daten vollkommen unstrukturiert, da selbst Texte einer vorgegebenen Grammatik der jeweiligen Sprache folgen. Der Begriff darf darum nicht zu streng genommen werden, vor allem da auch semistrukturierte Dokumente, die nur teilweise Freitext beinhalten, unter die Definition von Information Retrieval fallen.  Das Gegenteil stellen strukturierte Daten wie beispielsweise Datenbanken dar, die f�r einen Computer leicht zu verarbeiten sind. �blicherweise liegen die Sammlungen auf dem Computer gespeichert vor (\cite{Manning:08}, S.1-2). \\

\subsection{Unterschied zur Datenbankensuche}
Zum besseren Verst�ndnis hilft eine Abgrenzung zur Datenbanksuche, denn in Datenbanken liegen die Daten strukturiert in Form von Werttupeln bekannten Datentyps vor, was Definition \ref{Def} widerspricht.
Ein wesentliches Unterscheidungsmerkmal zwischen Information Retrieval und Datenbanksuche ist, dass bei der Datenbankensuche nicht mit vagen Anfragen umgegangen werden kann: Es kann zwar nach $(Miete < 300)$ gesucht werden, aber mit \textit{\glqq g�nstige Miete\grqq{}} w�re die Datenbank �berfordert: Wie ist g�nstig zu interpretieren? 
Ein Information-Retrieval-System kann hingegen solche Anfragen mit nicht genau definierter Bedeutung verarbeiten (\cite{Ferber:03}, S.10-11).


\section{Beispiel Websuche}
An dieser Stelle soll ein bekanntes Beispiel f�r Information Retrieval zur Veranschaulichung gegeben werden. 
Nahezu jeder benutzt im Alltag Web-Suchmaschinen.
Die Websuche stellt einen typischen Fall von Information Retrieval dar, was durch die Anwendung von Definition \ref{Def} deutlich wird: Hier sollen Freitext beinhaltende, d.h. unstrukturierte Dokumente (z.B. im HTML- oder pdf-Format) innerhalb des World Wide Webs aufgefunden werden, um das Informationsbed�rfnis des Internetnutzers zu befriedigen (\cite{Ferber:03}, S.6).
Relevante Suchergebnisse sind demnach Dokumente, welche die gesuchte Information beinhalten. Diese l�sst sich, wie in Abschnitt \ref{info} beschrieben, in drei Teile zerlegen, wobei der semantische Teil die Herausforderung f�r das Information-Retrieval-System darstellt. \\
Hierzu ein spezifisches Beispiel:
M�chte der Nutzer demn�chst seinen Urlaub in Kreta verbringen, k�nnte seine Suchanfrage \textit{\glqq Hotel g�nstig Kreta\grqq{}} lauten.
Der pragmatische Teil besteht darin, den Urlaub zu planen. Der syntaktische Teil ist ebenfalls leicht zu bestimmen: Die gesuchten Begriffe oder hierzu verwandte W�rter m�ssen in den Dokumenten auftauchen.
Als schwierig gestaltet sich hingegen der semantische Teil: Die Inhalte der Resultate m�ssen mit der urspr�nglichen Intention des Nutzers �bereinstimmen. Diese ist allerdings vage formuliert: Der Begriff \glqq g�nstig\grqq{} ist nicht n�her definiert. Nur ein Teil der Hotels, welche in der Ergebnisliste erscheinen, werden mit den Anspr�chen des Nutzers �bereinstimmen, vielleicht auch gar keine. Ein gutes Information-Retrieval-System zeichnet sich durch einen m�glichst gro�en Anteil relevanter Resultate unter allen zur�ckgelieferten Dokumenten aus. \\
H�ufig passiert es, dass zwar der syntaktische Teil erf�llt ist, d.h. die Suchbegriffe tauchen zwar im Dokument auf, allerdings stimmt der Kontext nicht mit dem Informationsbed�rfnis des Nutzers �berein. Dieses Problem tritt bei der Datenbanksuche, wo es keinerlei Interpretationsfreiraum gibt, gar nicht erst auf. 



\section{Bezug zur Problemstellung}
Dieser Abschnitt soll erkl�ren, inwiefern es sich bei der gegebenen Problemstellung um ein Information-Retrieval-Problem handelt.
Die Aufgabe besteht kurz gefasst darin, nach vom Nutzer ausgew�hlten, logisch verkn�pften Kriterien innerhalb eines Dokumentenarchivs zu suchen (siehe Abschnitt \ref{Problemstellung}). 
Damit ist die Definition \ref{Def} erf�llt, da hier Materialien innerhalb einer Sammlung, dem Dokumentenarchiv, aufgefunden werden sollen, um ein Informationsbed�rfnis zu befriedigen. \\
Dieses Bed�rfnis unterscheidet sich nat�rlich von Anfrage zu Anfrage, besteht aber allgemein gefasst darin, Dokumente wiederzufinden, z.B. eine bestimmte E-Mail. \\

\subsection{Teilweise strukturierte Daten}
Eine Besonderheit der Problemstellung ist hierbei, dass die Dokumente teilweise strukturiert sind, d.h. es liegt zwar Freitext vor, was mit Definition \ref{Def} �bereinstimmt, aber zus�tzlich sind strukturierte Metadaten vorhanden. 
Im Falle der Freitextsuche l�sst sich aufgrund der unstrukturierten Textform eindeutig von Information Retrieval sprechen, anders sieht es hingegen bei den Metadaten aus, welche die folgende Syntax und damit Struktur besitzen: \\

(Name Inhalt)\\

Es liegt dennoch ein Information-Retrieval-Problem vor, da der Begriff auch die Suche in teilweise strukturierten oder semistrukturierten Dokumenten einschlie�t (\cite{Manning:08}, S.1-2). 
Genau betrachtet sind selbst die Metadaten nicht vollkommen strukturiert: Der Datentyp des Inhalts ist offen gelassen und es gibt keinerlei Vorgaben, welche Metadaten in den Dokumenten auftreten m�ssen.  \\
In den folgenden Kapiteln wird beschrieben, auf welche Weise ein Information-Retrieval-System, das die gegebene Problemstellung l�st, konzipiert und realisiert werden kann.
 Hierzu werden zun�chst die hierf�r ben�tigten Kenntnisse �ber die beiden klassischen Information-Retrieval-Verfahren boolesches Retrieval (siehe Kapitel \ref{bool}) und Vektorraummodell (siehe Kapitel \ref{vector}) vermittelt.
 









\chapter{Boolean Information Retrieval}
\chapter{Das Vektorraum-Modell}


\section{Funktionsprinzip}
\section{Term Frequency}
\section{Document Frequency}
\section{Inverted Document Frequency}
\section{$Tf \times idf$ Weighting}
\subsection{Formeln}

\section{\"Ahnlichkeitsfunktion}
\subsection{Euklidische Distanz}
\subsection{Cosine Similarity}
\subsection{Alternativen}

\chapter{Sprachinterpreter}
\chapter{Implementierung}
In diesem Kapitel wird beschrieben, auf welche Weise das Information Retrieval System dieser Arbeit realisiert wurde.

\section{Teilweise strukturierte Dokumente}
Besonderheit der Problemstellung ist das Vorliegen der Dokumente in semistrukturierter Form (siehe \ref{Problemstellung}. \\
 Dies bedeutet, dass zwei unterschiedliche Teilprobleme zu l�sen sind: Zum einen die Keywordsuche, welche sich auf die Suche in strukturierten Metadaten bezieht und zum anderen die Freitextsuche. Es liegt nahe, beides getrennt zu l�sen, da die Suchen unterschiedliche Anforderungen besitzen.\\
 Bevor erkl�rt wird, wie die beiden Verfahren jeweils realisiert wurden, ist es wichtig, zun�chst eine Vorstellung zu haben, wie die zu durchsuchenden Dokumente des Archivs beschaffen sind, weshalb Abbildung \ref{example} ein Beispiel zeigt.
 \begin{lstlisting}[language=lisp, caption={Beispieldokument}
 \label{example},
	language=lisp]
 
 
 (absender ("<MaxMuster@muster-mail.de>"))
 (Betreff (" Umfrage"))
 (datum ("Wed, 22 Jun 2017 07:47:51 +0200"))
 (anzahlAnhaenge 0)
 (Termin nil)
 (ABSENDER "Max_Muster")
 
 (ABSENDER-MAIL-ADRESSE "MaxMuster@muster-mail.de")
 (EMPFAENGER ("doe>> John Doe"))
 (EMPFAENGER-MAIL-ADRESSEN ("johnd@muster-mail.de"))
 (BETREFF "Umfrage")
 (EMAIL-TYP "sent")
 (QUELLBOXART "SENT")
 
 Hallo John,
 
 ich werde dir die Umfragenformulare schnellstm�glich per Post 
 zukommen lassen.
 

 Viele Gr��e
 Max Muster
 
 \end{lstlisting}
 
 In der Praxis enthalten die Dokumente oft weitaus mehr Keywords, deren Inhalt sich auch �ber mehrere Zeilen erstrecken kann, sowie Kommentare, welche vom System als Freitext interpretiert werden. \\
 In diesem Beispiel handelt es sich zwar um eine E-Mail, die Keywords k�nnen jedoch inhaltlich vollkommen unterschiedlich ausfallen. Allen Dokumenten gemeinsam ist die 
 grundlegende Struktur aus \ref{struct}.
 
 \begin{lstlisting}[language=lisp, caption=Dokumentstruktur,label={struct}]
 	(Keywordname_1 Inhalt)
 	....
 	(Keywordname_n Inhalt)
 	
 	Freitext
 		
 \end{lstlisting}
 
 
 
 \section{Keywordsuche}
 
 \section{Freitextsuche}
\chapter{Zusammenfassung und Ausblick}

%In diesem Kapitel soll die Arbeit noch einmal kurz zusammengefasst werden. Insbesondere sollen die wesentlichen Ergebnisse Ihrer Arbeit herausgehoben werden. Erfahrungen, die z.B. Benutzer mit der Mensch-Maschine-Schnittstelle gemacht haben oder Ergebnisse von Leistungsmessungen sollen an dieser Stelle pr�sentiert werden. Sie k�nnen in diesem Kapitel auch die Ergebnisse oder das Arbeitsumfeld Ihrer Arbeit kritisch bewerten. W�nschenswerte Erweiterungen sollen als Hinweise auf weiterf�hrende Arbeiten erw�hnt werden.


\section{Zusammenfassung}

Im Kapitel Einleitung und Problemstellung wurde die Aufgabenstellung erl�utert, wobei sich bereits zeigte dass diese aufgrund des semistrukturierten Aufbaus der zu durchsuchenden Dokumente eine sehr spezifische L�sung erfordern w�rde. Die Unterteilung in Metadaten und Freitext sowie das logische Verkn�pfen der Suchbedingungen stellen die wesentlichen Herausforderungen dar.
Anschlie�end wurden die notwendigen theoretischen Kenntnisse vermittelt. Die Bedeutung des Begriffs Information Retrieval wurde erkl�rt und es zeigte sich, dass diese sehr weit gefasst ist,  weshalb diese Arbeit die ausgesprochen umfangreiche Thematik nur ansatzweise behandeln kann. Es wurden die beiden bekanntesten klassichen Information Retrieval  Verfahren vorgestellt; das boolesche Retrieval und das Vektorraummodell, sowie Methoden, anhand derer sich solche Modelle im Hinblick auf Qualit�t bewerten und Vergleichen lassen. Hierbei zeigte sich, dass eine solche Bewertung im Rahmen dieser Arbeit aufgrund des Aufwands leider nicht durchf�hrbar ist. \\
Im Implementierungskapitel wurde deutlich, dass die Verwendung eines der beiden klassischen Verfahren nicht ausreicht, die Kombination aus booleschen Retrieval und Vektorraummodell jedoch hervorragend passt. Das boolesche Retrieval musste allerdings geringf�gig modifiziert werden, indem nicht nur das Vorkommen der Attribute in den Dokumenten vermerkt wird, sondern auch deren Inhalt inklusive Datentyp. Beim Vektorraummodell zeigte sich, dass sich Speicher sparen l�sst, indem alle Elemente des Vektors, welche eine 0 enthalten, nicht eingetragen werden. \\
Zuletzt zeigte sich, dass die Benutzeroberfl�che eine nicht zu untersch�tzende Herausforderung darstellt,
da der Anwender sein Informationsbed�rfnis auf m�glichst unkomplizierte Art ausdr�cken k�nnen soll. Ein Information Retrieval System muss vom Nutzer auch verstanden werden. Es wurde sich daf�r entschieden, m�glichst einfache UI-Elemente wie CheckBoxes zu w�hlen sowie die Oberfl�che in drei thematische Bereiche zu gliedern, um die notwendige �bersicht zu bieten.




 \section{Ausblick: W�nschenswerte Erweiterungen}
Einige Erweiterungen konnten aus zeitlichen Gr�nden in der Implementierung nicht umgesetzt werden, sind jedoch w�nschenswert. \\
Die Im Kapitel Grundbegriffe beschriebene Lemmatisierung (siehe Abschnitt \ref{lemmatisierung}) w�rde eine gro�e Verbesserung hinsichtlich des Speicherbedarfs bedeuten, da weniger Terme gespeichert werden m�ssen. Zudem w�rde das Erg�nzen um Lemmatisierung auch die Suche verbessern, da auch Dokumente mit zum Suchbegriff verwandten W�rtern als Treffer gewertet werden. \\
Weiterhin w�nschenswert ist die Erg�nzung um Spelling Correction, d.h. um eine Rechtschreibkorrektur die �hnliche Suchbegriffe vorschl�gt, falls der Anwender sich bei der Eingabe vertippt hat. Dies w�rde die Anwenderfreundlichkeit des Systems erheblich steigern, insbesondere in F�llen in welchen der Anwender seinen Tippfehler nicht bemerkt und unerwartet keine Resultate erh�lt. Da Nutzer in der Regel Spelling Correction aus der allt�glichen Websuche gew�hnt sind, ist es w�nschenswert auch dieses Information Retrieval System damit auszustatten.


%-Stopwords \\
%-Lemmatisierung \\
%-Spelling Correction \\

% ...
%--------------------------------------------------------------------------
\backmatter                        		% Anhang
%-------------------------------------------------------------------------
\bibliographystyle{geralpha}			% Literaturverzeichnis
\bibliography{literatur}     			% BibTeX-File literatur.bib
%--------------------------------------------------------------------------
\printindex 							% Index (optional)
%--------------------------------------------------------------------------
\begin{appendix}						% Anh�nge sind i.d.R. optional
   \chapter{Glossar}

\abbreviation{DisASTer}		{DisASTer (Distributed Algorithms Simulation Terrain), A platform for the Implementation of Distributed Algorithms}
\abbreviation{DSM}			{Distributed Shared Memory}
\abbreviation{AC}			{Linearisierbarkeit (atomic consistency)}
\abbreviation{SC}			{Sequentielle Konsistenz (sequential consistency)}
\abbreviation{WC}			{Schwache Konsistenz (weak consistency)}
\abbreviation{RC}			{Freigabekonsistenz (release consistency)}
			% Glossar   
   \chapter{Erkl�rung der Kandidatin / des Kandidaten}

\begin{description}[$\Box$~]
\item[$\Box$] Die Arbeit habe ich selbstst�ndig verfasst und keine anderen als die angegebenen Quellen- und Hilfsmittel verwendet.\\

\item[$\Box$] Die Arbeit wurde als Gruppenarbeit angefertigt. Meine eigene Leistung ist\\
...\\

Diesen Teil habe ich selbstst�ndig verfasst und keine anderen als die angegebenen Quellen und Hilfsmittel verwendet. \\

Namen der Mitverfasser: ...

\end{description}

\vspace{2cm}

\begin{minipage}[t]{3cm}
\rule{3cm}{0.5pt}
Datum
\end{minipage}
\hfill
\begin{minipage}[t]{9cm}
\rule{9cm}{0.5pt}
Unterschrift der Kandidatin / des Kandidaten
\end{minipage}	% Selbstst�ndigkeitserkl�rung
\end{appendix}

\end{document}
